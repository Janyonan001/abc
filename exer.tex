\documentclass[twoside]{ctexart}
\usepackage{color}
\usepackage{tikz}
\usetikzlibrary{positioning,trees}
\usepackage[margin=2.5cm,a4paper]{geometry}
\usepackage{amsmath,amssymb}
\setmainfont{SimSun}[BoldFont={Microsoft YaHei}, ItalicFont={KaiTi}]
%\setsansfont{Arial}[BoldFont={Arial Bold}, ItalicFont={Arial Italic}]
%\setCJKmainfont{SimSun}[BoldFont=SimHei, ItalicFont=KaiTi]
\usepackage{fancyhdr}
\pagestyle{fancy}
\usepackage{exam-zh-choices}
\setchoices{
	max-columns=4,
	top-sep=0.5em,
	linesep=0.5em}

\ctexset{
	section/name={第,讲},
	section/number=\chinese{section},
	section/format=\huge\bfseries\centering,
	section/beforeskip=0pt,
	subsection={name={【考点,】},
		number=\chinese{subsection},
		format=\raggedright\zihao{4}\heiti,
		beforeskip=0pt},
	subsubsection={
		name={【题型,】},
		format=\raggedright\zihao{-4}\heiti,
		number=\chinese{subsubsection},
		}
	}
%定义例题与变式训练可用环境形式或用命令形式
\newcounter{exam}[subsection]
\newcounter{examp}[exam]
\newcounter{exampl}[exam]
\newcounter{exampll}[examp]
\newenvironment{liti}[1]{\stepcounter{exam}\noindent\zihao{-4}\songti
	{【例\theexam】 #1}}{}
\newenvironment{bian}[1]{\stepcounter{examp}\noindent\zihao{-4}\songti
	{变式训练\theexam-\theexamp.} #1}{}
\newcommand{\litii}[1]{\stepcounter{exam}\noindent\zihao{-4}\songti
	{【例\theexam】#1}}
\newcommand{\bianshi}[1]{\stepcounter{examp}\noindent\zihao{-4}\songti
	{变式训练\theexam-\theexamp.} #1}
\newcommand{\litx}[1]{\stepcounter{exampl}\songti\fontsize{12}{20}\selectfont
	{(\theexampl)} #1}
\newcommand{\bianx}[1]{\stepcounter{exampll}\songti\fontsize{12}{20}\selectfont
		{(\theexampll)} #1}
\usepackage{verbatim}	
\usepackage{enumitem}
\newlist{kaoqing}{description}{1}
\newlist{zhishihuigu}{enumerate}{2}
%before= 可以对列表项内容进行设置。使用\fontsize后,一定要加上\selectfont使设置生效。
\setlist[kaoqing, 1]{format=\color{blue}\heiti\zihao{-4}, leftmargin=0pt,itemindent=3em,
	before=\itshape\fontsize{14}{16}\selectfont\color{red!40!black},after=\normalsize}

\setlist[zhishihuigu, 1]{leftmargin=0pt,itemindent=4em,labelsep=0.5em,labelwidth=1.5em,align=right,
	label={\arabic*.},before=\fontsize{12}{14}\selectfont,after=\normalsize,}

\setlist[zhishihuigu, 2]{leftmargin=2em,itemindent=3.5em,labelsep=0.5em,label=(\arabic*),
	format=\color{violet}\heiti\zihao{-4},
	before=\color{cyan!80!black}\itshape}

\newcommand{\bdtext}[1]{\noindent\fcolorbox{black}{black}{\heiti\zihao{4}\color{white}#1}}
\newcommand{\gongsh}[1]{$\displaystyle #1 $}
\begin{document}

	\section{函数的三要素}
	\bdtext{知识框图}
%\iffalse
	\begin{center}
	\begin{tikzpicture}[grow'= east,
		root node/.style={shape=rectangle,rounded corners,fill=blue!20,font=\zihao{3}\heiti,text=black, inner sep=10pt,draw=blue,very thick},
		level 1 child node/.style={shape=rectangle,rounded corners,font=\heiti,draw=blue,thick,fill=blue!10,inner sep=5pt},
		every child node/.style={shape=rectangle,rounded corners,draw=blue,thick,inner sep=5pt,anchor=west},
		level 1/.style={level distance=2cm,sibling distance=3.2cm,},
		level 2/.style={level distance=2cm,sibling distance=2em,font=\kaishu},
		edge from parent/.style={draw=red,out=0,in=180}]
		
		\node[root node] (a) {函数三要素} 
			child {node[level 1 child node](b) {考点一:定义域}
				child {node {具体函数的定义域}}
				child {node {已知定义域求参数范围}}
				child {node {抽象函数的定义域}}}
			child {node[level 1 child node] (c) {考点二:解析式}
				child {node {待定系数法}}
				child {node {换元法}}
				child {node {配凑法}}
				child {node {方程组法}}
				child {node {迭代法}}}
			child {node[level 1 child node] (d) {考点三:值域}
				child {node{二次函数模型}}
		  		child {node {分式函数型}}
				child {node {根式函数型}}} ;
		\end{tikzpicture}
	\end{center}
%\fi	
\begin{comment}
			\begin{tikzpicture}[every to/.style={out=0,in=180},]
			\node(a){函数三要素};
			\node[above  =5cm of c ] (b) {考点一:定义域};
				\node[right= of b] (ba) {已知定义域求参数的范围};
				\node [above=  of ba ] (bb) {具体函数的定义域};
				\node [below=  of ba ] (bc) {抽象函数的定义域};
			\node[right=2em of a] (c) {考点二:解析式};
				\node[right=of c] (cc) {配凑法};
				\node[above= of cc] (cb) {换元法};
				\node[above= of cb] (ca) {待定系数法};
				\node[below= of cc] (cd) {方程组法};
				\node[below= of cd] (ce) {迭代法};
		\node[below =3cm of c ] (d) {考点三:值域};	
		\draw[->] (a) to (b);
		\draw[->] (a) to (c);
		\draw[->] (a) to (d);
		\draw[->] (b) to (ba);
		\draw[->] (b) to (bb);
		\draw[->] (b) to (bc);
		\draw [->] (c) to (ca)
		(c) to (cb)
		(c) to(cc)
		(c) to (cd)
		(c) to (ce);
	\end{tikzpicture}
\end{comment}
	\bdtext{考情分析}
	\begin{kaoqing}
		\item [【考察重点】]函数求解析式,函数求值域。
		\item [【考察题型及能力】]函数求定义域,解析式常见于小题,较为简单,5分。函数求值域经常和其它知识综合考察,较难。
		\item [【考察能力】]抽象概括能力,数形结合能力。
	
		在高考中,函数知识是必考知识,也是跟其它知识链接最密切的内容,函数定义域是函数内容的基础,是未来讲解一切知识的前提,所以必须分类详细,讲解到位。
	\end{kaoqing}		
		
	\bdtext{知识回顾}
	\begin{zhishihuigu}
		\item 函数是一种非空的数集组成的映射,是从自变量$x$到应变量$y$的对应关系;其中$x$的范围叫做定义域;
		\item 定义域的常见形式有分式,根式,指数,对数,复合函数以及抽象函数;具体函数定义域的常见类型:
		\begin{zhishihuigu}
			\item 分式中分母不为零			
			\item 偶次根式非负
			\item 零次幂底数非零
			\item 对数函数底数大于零
			\item 正切函数$\displaystyle \left\lbrace x \neq k\pi+\frac{\pi}{2},x\in R \right\rbrace $
			\item 当题中出现多个函数的四则运算及复合时,注意考虑每一个函数定义域并取交集
		\end{zhishihuigu}
		\item 抽象函数常见类型
		\begin{zhishihuigu}
			\item 已知$f(x)$定义域求$f(f(x))$定义域
			\item 已知$f(g(x))$定义域求$f(x)$定义域
			\item 已知$f(g(x))$定义域求$f(h(x))$定义域
		\end{zhishihuigu}
		\item 函数定义域应用:根据函数定义域求参数范围问题
	\end{zhishihuigu}

%\begin{kaodian}
\subsection{函数的定义域}
\subsubsection{具体函数的定义域}
		
		\litii{求函数$\displaystyle y=\frac{3x+1}{2x^2-1}$的定义域}\vfill
		\bianshi{求函数$\displaystyle y=\frac{1}{x+\frac{2}{x+\frac{2}{3}}}$的定义域\vfill}
		\litii{求函数$y=\sqrt{\log_2{(3x-1)}}$的定义域\vfill}
		\bianshi{求函数$y=\sqrt{-x^2+36}+\ln(\sin x)$的定义域\vfill}
		
		\null\newpage
		\litii{求函数$\displaystyle y=\left(\frac{x-2}{2x+1}\right)^0$的定义域\vfill}
		\bianshi{求函数$\displaystyle {f(x)=(x-2)^0+\frac{1}{\sqrt{x-1}}}$定义域}\vfill
		
		\litii{	求函数$y=\tan (3x)$的定义域\vfill}		
		\bianshi{求函数$\displaystyle y=\tan (\frac{\pi}{4}-2x)$的定义域}\vfill
	
		\litii{	(2012山东文3)函数$\displaystyle f(x)=\frac{1}{x+1}+\sqrt{4-x^2}$的定义域为(   )}
		\begin{choices}
			\item $\left[-2,0\right)\bigcup\left(0,2\right] $ 
			\item $\left(-1,0\right)\bigcup\left(0,2\right]$
			\item $[-2,2]$
			\item $\left(-1,2\right]$
		\end{choices}\vfill
		\bianshi{(2011江西理3)若$f(x)=\frac{1}{\log_{\frac{1}{2}}(2x+1)}$,则$f(x)$的定义域为(  )}
		\begin{choices}
			\item $\left(-\frac{1}{2},0\right)$
			\item $\left(-\frac{1}{2},0\right]$
			\item $\left(-\frac{1}{2},+\infty\right)$
			\item $(0,+\infty)$
		\end{choices}
		\newpage
%%%%%%%%%%%%%%%%%%%%%%%%%%%%%%%%%%%%%%%%%%%%%%%%%%%%%%%%%%%%%%%%%%%%%%%%%%%%%%%%	
	\subsubsection{已知定义域求参数范围}
	\litii{ 若函数$\displaystyle f(x)=\frac{1}{x^2+4ax+3}$的定义域是$R$,求实数$a$的取值范围}\vfill
	\bianshi{若函数$\displaystyle f(x)=\frac{1}{ax^2+4ax+3}$的定义域是$R$,求实数$a$的取值范围}\vfill
	
	\litii{若函数$\displaystyle f(x)=\sqrt{2^{x^2+2ax-a}-1}$的定义域是$R$,求实数$a$的取值范围。}\vfill
	\bianshi{若函数$\displaystyle y=\sqrt{\left(a^2-1\right)x^2+(a-1)x+\frac{2}{a+1}}$的定义域为$R$,求实数$a$的范围。}\vfill
	
	\litii{已知函数$ \displaystyle y=\lg\left(ax^2-ax+1\right)$的定义域为$R$,求$a$的范围。}\vfill
	\bianshi{已知函数$\displaystyle y=\lg\left(ax^2-ax+1\right)$的定义域是$R$,求$a$的范围。}\vfill
	\subsubsection{抽象函数的定义域}
	\litii{已知函数$f(x)$的定义域是$\left[1,2\right]$,求$f(2x-1)$的定义域。}\vfill
	\bianshi{已知函数$f(x-1)$的定义域是$\left(2,3\right)$,求$f(x^2)$的定义域。}\vfill
	
	\litii{已知函数$f(x^2-1)$的定义域是$(2,3)$,求$f(2x^2)$的定义域。}\vfill
	\bianshi{已知函数$f(2^x)$的定义域为$\left[1,2\right]$,求$f(\log_2x)$的定义域。}
	%%%%%%%%%%%%%%%%%%%%%%%%%%%%%%%%%%%%%%%%%%%%%%%%%%%%%%%%%%%%%%%
	\newpage
	\subsection{函数的解析式}
	\litii{(2015-2016合肥六中第一次阶段考18)已知二次函数$f(x)$满足$f(1)$=0,且$f(x+1)-f(x)=4x+3$,求$f(x)$的解析式。}\vfill
	\bianshi{(2016合肥一模)若函数$f(x)$是幂函数,满足$\frac{f(4)}{f(2)}=3$,则$f(x)$的解析式为\underline{\hspace{4em}}.}\vfill
	
	\litii{已知$f(\sqrt{x}+1)=x+2\sqrt{x}$,求$f(x)$的解析式。}\vfill
	\bianshi{已知函数$g(x)$满足$g(2x-1)=x^2-x$,求$g(x)$的解析式。}\vfill
	
	\litii{已知$\displaystyle f(x-\frac{1}{x})=x^3-\frac{1}{x^3}$,求$y$的解析式。}\vfill
	\bianshi{(2016-2017合肥六中期中考试7)函数$\displaystyle f(x-\frac{1}{x})=x^2+\frac{1}{x^2}$,则$f(x)=\underline{\hspace{4em}}$.}\vfill
	
	\litii{已知函数$f(x)$满足$f(x)-2f(\frac{1}{x})=x^2-2x$,求$f(x)$的解析式。}\vfill
	\bianshi{已知函数$f(x)$为奇函数,$g(x)$为偶函数,满足$f(x)-2g(x)=x^2-2x$,分别求$f(x)$,$g(x)$的解析式。}\vfill
	
	\litii{(2014-2015合肥六中期中考试14)定义在$R$上的函数$f(x)$满足$f(x+1)=-2f(x)$,若当$0\leqslant x \leqslant1$时,$f(x)=x-x^2$,则当$-1\leqslant x \leqslant 0$时,$f(x)=$\underline{\hspace{4em}}.}\vfill
	\bianshi{(2014-2015合肥八中期中考试19)定义域为$R$的函数$f(x)$满足$f(x+2)=3f(x)$,当$x \in [0,2]$时,$f(x)=x^2-2x$,当$x \in [-4,-2]$时,求$f(x)$的解析式。}\vfill\null
	%%%%%%%%%%%%%%%%%%%%%%%%%%%%%%%%%%%%%%%%%%%%%
	\newpage
	\subsection{函数的值域}
	\subsubsection{二次函数型}
	
	\litii{求函数$\displaystyle y=\sqrt{\frac{1}{1-x(1-x)}}$的值域。}\vfill
	\bianshi{求函数$y=\sqrt{-x^2+4x+5}$的值域。}\vfill
	
	\litii{求函数$y=4^x+2^{x+1}+1$的值域。}\vfill
	\bianshi{求函数$y=3\cdot4^x-2^x+3,x \in [-1,2]$的值域。}\vfill
	
	\litii{求函数$f(x)=\log_2\sqrt{x}\cdot \log_{\sqrt{2}}{(2x)}$的最小值。}\vfill
	\bianshi{已知函数$f(x)=3+\log_2x,x \in [1,4]$,则$g(x)=f(x^2)-[f(x)]^2$的值为(  )}
	\begin{choices}
		\item [-18,-2]
		\item [-11,-6]
		\item [-18,6]
		\item [-1,-2]
	\end{choices}\vfill
	
	\litii{求$y=\sin x+ \cos^2x-1 $的值域。}\vfill
	\bianshi{求函数$y=\sin x+\cos x +\sin x \cos x+1$的值域。}\vfill
\newpage
%%%%%%%%%%%%%%%%%%%%%%%%%%%%%%%%%%%%%%%%%
	\subsubsection{分式函数型}
	\litii{求下列函数的值域:}
	
	\litx{$\displaystyle f(x)=\frac{2x-1}{x+1}$}\hfill
	\litx{$\displaystyle f(x)=\frac{e^x-1}{e^x+1}$}\hfill\null
	\vfill
	\bianshi{求下列函数值域:}
	
	\bianx{$\displaystyle f(x)=\frac{2x-1}{x+1}(x\geq1)$}\hfill
	\bianx{$\displaystyle f(x)=\frac{2\sin x-1}{3 \sin x+2}$}\hfill\null
	\vfill
	
	\litii{求函数$\displaystyle f(x)=\frac{2x^2+5x+10}{x+1}, x \in [2,3]$的值域。}\vfill	
	\bianshi{求函数$\displaystyle y=\frac{2x+1}{x^2-2x+1}$的值域。}\vfill
	
	\subsubsection{根式函数型}
	\litii{求函数$f(x)=-2x+\sqrt{x-1}$的值域}\vfill
	\bianshi{求函数$f(x)=-2x-\sqrt{-x+1}$的值域}\vfill
	
	\litii{函数$f(x)=2x-\sqrt{x-1}$的值域是(  )}
	\begin{choices}
		\item $\left[0,+\infty\right)$
		\item $\left[\frac{17}{8},+\infty\right)$
		\item $\left[\frac{5}{4},+\infty\right)$
		\item $\left[\frac{15}{8},+\infty\right)$
	\end{choices}
	\vfill
	\bianshi{求函数$y=\sqrt{x-1}-x$的值。}\vfill
	
	\litii{求函数$y=\sqrt{x+3}+\sqrt{5-x}$的值域}\vfill
	
	\bianshi{求函数$y=\sqrt{x-3}+\sqrt{4-x}$的值域}\vfill
	
	\litii{求函数$y=\sqrt{x+1}-\sqrt{x-1}$的值域}\vfill
	
	\bianshi{求函数$y=\sqrt{2x-1}-\sqrt{2x-2}$的值域}\vfill
	
\end{document}